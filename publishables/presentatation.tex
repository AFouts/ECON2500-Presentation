% ||||||||||||||||||||||||||||||||||||||||||||||||||||||||||||||||||||
\documentclass{beamer}
\usepackage[utf8]{inputenc}
\usepackage[english]{babel}
\usepackage{amsmath}
\usepackage{setspace}
\usepackage{bm}
\usepackage{graphicx}
\usetheme{CambridgeUS}
\usepackage{stmaryrd}
\usepackage{mathrsfs}
\usepackage[makeroom]{cancel}
\usepackage{amsmath,amsthm,stmaryrd,amssymb,bbm,amsfonts,amstext, graphicx, multicol, array}



\newcommand\vertarrowbox[3][2ex]{%
  \begin{array}[t]{@{}c@{}} #2 \\
  \left\uparrow\vcenter{\hrule height #1}\right.\kern-\nulldelimiterspace\\
  \makebox[0pt]{\scriptsize#3}
  \end{array}%
}

% ||||||||||||||||||||||||||||||||||||||||||||||||||||||||||||||||||||
% ||||||||||||||||||||||||||||||||||||||||||||||||||||||||||||||||||||
\title{ECON2840 Spring 2024\\Paper Presentation}
\subtitle{Daron Acemoglu, The simple macroeconomics of AI, Economic Policy, Volume 40, Issue 121, January 2025, Pages 13–58}
\author{Adrien Foutelet}
\date{\today}
% ||||||||||||||||||||||||||||||||||||||||||||||||||||||||||||||||||||
\AtBeginSection[]{
\begin{frame}
\vfill
\centering
\begin{beamercolorbox}[sep=8pt,center,shadow=true,rounded=true]{title}
\usebeamerfont{title}\insertsectionhead\par%
\end{beamercolorbox}
\vfill
\end{frame}
}
% ||||||||||||||||||||||||||||||||||||||||||||||||||||||||||||||||||||


% ||||||||||||||||||||||||||||||||||||||||||||||||||||||||||||||||||||
% ||||||||||||||||||||||||||||||||||||||||||||||||||||||||||||||||||||
% ||||||||||||||||||||||||||||||||||||||||||||||||||||||||||||||||||||
\begin{document}
% ||||||||||||||||||||||||||||||||||||||||||||||||||||||||||||||||||||
% ||||||||||||||||||||||||||||||||||||||||||||||||||||||||||||||||||||
% ||||||||||||||||||||||||||||||||||||||||||||||||||||||||||||||||||||
%
\maketitle
% ||||||||||||||||||||||||||||||||||||||||||||||||||||||||||||||||||||
% ||||||||||||||||||||||||||||||||||||||||||||||||||||||||||||||||||||
%
\section{Context and Motivation}


\begin{frame}{A lot of hype, hindisght from robotization, first experimental results}

    \begin{itemize}
        \item In the media:
        \begin{itemize}
            \item Science fiction fantasies,
            \item ChatGPT: fastest spreading tech mlatform in history,
        \end{itemize}
        \item In the business world:
        \begin{itemize}
            \item Goldman Sachs (2023): Over ten years, 7\% increase in global GDP, 1.5\% annual increase in US productivity growth,
            \item McKinsey Global Institute (2023): a USD 17-26 trillion increase to the global economy.
            \item 12\% of the labor force.
        \end{itemize}
        \item Recall robotization (Acemoglu, Restrepo, 2020):
        \begin{itemize}
            \item Positive outcomes on firm owners and managers,
            \item Negative outcomes on workers.
        \end{itemize}
        \item Experiments: nontrivial productivity gains driven by improvements for less productive/lower-performing workers.
    \end{itemize}

\end{frame}

\begin{frame}{Research question and strategy}

\textit{What insight can we get, from the Macro labor toolbox, on the outcomes of AI for a ten-year horizon?}
\newline
\begin{itemize}
    \item Fitting the Task.Skill.Capital framework from Acemoglu, Autor (2011); Restrepo (2018, 2019, 2022),
    \item Empirical calibration and prediction.
\end{itemize}

\end{frame}

\section{Model}

\begin{frame}{Production}

\begin{itemize}
    \item A unique final good,
    \item All markets are competitive,
    \item A continuum of tasks: \([0;N]\),
    \item Output of task z: \(y(z)\),
    \item Production function: \textcolor{red}{\(Y=B(N)\Big(\int_{0}^{N}y(z)^{\frac{\sigma -1}{\sigma}}dz\Big)^{\frac{\sigma}{\sigma -1}}\)},
    \item Constant elasticity of substitution (tasks are complements): \(\sigma\),
    \item System-wide effect of new tasks: \(B(N)\),
    \item Task production function: \textcolor{red}{\(y(z)=A_L\gamma_L(z)l(z)+A_K\gamma_Kk(z)\)}
    (in fact \(y(z)=A_L\gamma_L(z)[l(z)^{1-\kappa}k_C(z)^\kappa)]+A_K\gamma_Kk(z)\)),
    \item labor/capital-augmenting productivity terms: \(A_L\), \(A_K\),
    \item Labor/capital task-specific productivity schedules: \(\gamma_L(z)\), \(\gamma_K(z)\).
\end{itemize}
Assume \(\frac{\gamma_L(z)}{\gamma_K(z)}\) increasing in z.

Hence labor has a comparative advantage in higher-indexed tasks.

\textcolor{red}{Hence \(z\in[0;I]\) is produced via capital; \(z\in]I;N]\) is produced via labor.}

\end{frame}{}

\begin{frame}{Labor, capital, consumption}
 
\begin{itemize}
    \item Measure-one population,
    \item High and low skill labor for fractions \(\Phi^H\), \(\Phi^U\) of the population endowed with \(\lambda^H\), \(\lambda^U\) units of labor (tasks can be performed by either),
    \item \(\lambda^U < \lambda^H\),
    \item Supply of labor: \textcolor{red}{\(\Phi^H\lambda^H+\Phi^U\lambda^U\)=L},
    \item Wage rate: \(\omega\),
    \item Aggregated capital: \(K=\int_{0}^{N}k(z)dz\)
    \item Task-specialised capital \(k(z)\) produced linearily from final good Y with price \textcolor{red}{\(R(z)=R(K)\rho(z)\)},
    \item \textcolor{red}{Households consume the final good net of capital expenditures}.
\end{itemize}    

\end{frame}

\begin{frame}{Equilibrium}

Characterization of an equilibrium:

\begin{itemize}
    \item z is produced by labor iff:
    \[\frac{\omega}{A_L\gamma_L(z)}<\frac{R(z)}{A_K\gamma_K(z)}\],
    \item
    \[k(z)=argmax_{k(z)} \{Y-R(z)k(z)\}\],
    \item Labor market clearing:
    \[L=\int_{0}^{N}l(z)dz\].
\end{itemize}
We skip the solution. The point is to compute comparative statics.

\end{frame}

\begin{frame}{Towards comparative statics: how AI affects production}

\begin{itemize}
    \item Extensive margin automation i.e., increase in \(I\):
    \begin{itemize}
        \item AI reduces the cost of capital \(\rho(z)\) for some marginal tasks above \(I\),
        \item AI increases the effectiveness of capital \(\gamma_K(z)\) for some marginal tasks above \(I\),
    \end{itemize}
    \item Greater task complementarities, for \(z\in[0;I]\):
    \begin{itemize}
        \item AI raises the task-specific productivity of labor \(\gamma_L(z)\),
        \item AI reduces the cost of complementary capital \(k_C(z)\),
    \end{itemize} 
    \item Intensive margin automation, for \(z\in[0;I]\):
    \begin{itemize}
        \item AI reduces the cost of capital \(\rho(z)\),
        \item AI increases the effectiveness of capital \(\gamma_K(z)\),
    \end{itemize}
    \item New labor-intensive products or tasks i.e., increase in N:
    \begin{itemize}
        \item "Good" or "bad" in terms of social value, possibly causing externalities.
    \end{itemize}
\end{itemize}
    
\end{frame}

\begin{frame}{Comparative statics}

\begin{itemize}
    \item Ambiguous sign of \(\frac{dln\omega}{dI}\):
    \begin{itemize}
        \item Increase in productivity \(\rightarrow\) Increase in wages,
        \item Displacement of workers across tasks \(\rightarrow\) Decrease in wages.
    \end{itemize}
    \item Ambiguous sign of \(\frac{dln\omega}{d\gamma_L(z)}\):
    \begin{itemize}
        \item Increase in easiness to perform \(\rightarrow\) Decrease in task price not fully offset,
        \item Unclear offset by increase in productivity.
    \end{itemize}
    \item For \(z\in[0;I]\), \(\frac{dln\omega}{d\gamma_K(z)}>0\), \(\frac{dln\omega}{d\rho(z)}>0\), \(\frac{dN}{d\rho(z)}>0\).
\end{itemize}
\end{frame}

\begin{frame}{Simplifying the job for aggregates: Hulten's theorem}

How micro-level productivity improvements translate into macro changes.

\[GDP=Y=\int_{0}^{N}p(z)y(z)dz\]

At the equilibrium, any small change in an element of \(\{B;A_L;A_K;\gamma_L(z);\gamma_L(z);I;N\}\) (technology) has second-order impact on GDP via:
\begin{itemize}
    \item Reallocation of factors across tasks,
    \item Prices.
\end{itemize} 

This takes us to
\[dlnY=dlnTFP+S_KdlnK\]
(with \(s_K\) the capital share of GDP) and
\[dlnTFP=\text{Costs savings from automation}*\text{GDP share of AI-impacted tasks}\]

\end{frame}

\begin{frame}{Finalizing the theory}

\begin{itemize}
    \item Distinguishing between easy and hard to learn tasks
    
    \(\rightarrow\) Different productivity gain magnitudes,
    \item Welfare analysis including the negative exterality from "bad" tasks,
    \item Inclusion of multiple demographic groups
    
    \(\rightarrow\) Theoretical arguments in favour of rising inequality due to an increase in the productivity of low-skilled workers.
\end{itemize}

\end{frame}

\section{Empirics}
    
\begin{frame}{Logic, sources, first estimates}

    Estimates of which tasks can be automated with AI;
    Estimates of the cost savings;
    Estimates of new tasks.
    
    \(\rightarrow\)
    
    TFP growth estimates; GDP growth estimates; Wage growth estimates; Inequality implications.
    \newline
    \begin{itemize}
        \item GDP share of tasks impacted by AI within the next 10 years:
        \begin{itemize}
            \item Eloundou et al. (2023) \(\rightarrow\) Ask GPT-4 to classify O*NET 19,265 tasks and 2,087 Detailed Work Activity and infer an automation index.
            \item U.S. Bureau of Labor Statistics National Occupational Employment and Wage Estimates.
        \end{itemize}
        4.6\%.
        \item Cost savings from AI:
        \begin{itemize}
            \item Peng et al. (2023),
            \item Noy and Zhang (2023),
            \item Brynjolfsson et al. (2023).
        \end{itemize}
        20.5\%.

    \end{itemize}

\end{frame}

    
\begin{frame}{TFP growth, GDP growth, bad tasks}
    
    \begin{itemize}
        \item TFP growth over 10 years:
        \begin{itemize}
            \item Ignoring hard and easy tasks:
            
            0.71\%. Likely to reach 1\% if the cost of AI decreases optimisticly over time,
            
            \item Considering hard and easy tasks:
            
            0.55\%. Likely to reach 1\% if the cost of AI decreases optimisticly over time,   
        \end{itemize}
        \item GDP growth over 10 years:
        \begin{itemize}
            \item Ignoring hard and easy tasks:
            
            1.1\%,

            \item Considering hard and easy tasks:
            
            0.92\%,
        \end{itemize}
        \item Bursztyn et al. (2023) (estimate the welfare effect of social media):
        
        Currently 0.072\% of GDP, sizable effect on welfare.
    \end{itemize}

\end{frame}   
    
\begin{frame}{Wages and inequality}

Acemoglu and Restrepo (2022): The chore of the reasoning is the ripple effects: the displacement of one demographic group impacts others.

\begin{itemize}
    \item Least exposed: workers with less than high school,
    \item Medium: postgraduate workers,
    \item Most exposed: workers with college degrees and those with high school degrees or some college.
\end{itemize}

Within the next ten years:
\begin{itemize}
    \item Predicted wage impacts do not appear to increase inequality between education groups.
    \item Predicted impact on GDP considering hard and easy tasks: 1.62\%.
    \item Predicted impact on capital share considering hard and easy tasks: 0.38\%.
\end{itemize}

Consequences on inequality of AI will not be as adverse as those from pre-AI automation.

No evidence of inequality reduction.

\end{frame}       

\section{Conclusion}


\end{document}
% ||||||||||||||||||||||||||||||||||||||||||||||||||||||||||||||||||||
% ||||||||||||||||||||||||||||||||||||||||||||||||||||||||||||||||||||
% ||||||||||||||||||||||||||||||||||||||||||||||||||||||||||||||||||||